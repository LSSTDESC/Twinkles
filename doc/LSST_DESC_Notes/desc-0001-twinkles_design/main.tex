%
% ======================================================================
\RequirePackage{docswitch}
% \flag is set by the user, through the makefile:
%    make note
%    make apj
% etc.
\setjournal{\flag}

\documentclass[\docopts]{\docclass}

% You could also define the document class directly
%\documentclass[]{emulateapj}

\input{macros}

\usepackage{graphicx}
\graphicspath{{./}{./figures/}}
\bibliographystyle{apj}

%
% ======================================================================

\begin{document}

\title{ Twinkles: Science Goals, Survey Simulation and Analysis Design }

\maketitlepre

\begin{abstract}

Accurate cosmography demands a demonstration, on plausibly realistic mock data, that astrophysical input parameters can be recovered in an end-to-end test.
Twinkles is a project to partially fulfill this condition for the two LSST DESC time domain probes, type Ia supernovae and strong lens time delays.
We are simulating a 10-year multi-filter LSST sky survey of a tiny (c.\ 200 square arcmin) patch of sky that has been ``sprinkled'' with an over-abundance of supernovae and strong lenses, processing the images using LSST DM stack software and then analyzing the resulting catalogs.
During this process we aim to learn an ``error model,'' that defines the density function that could be used as the likelihood of the paramters given LSST cataloged flux measurements, and the sampling distribution in a catalog-level mock data generator.
We envision the Twinkles project having at least two phases, corresponding to the DC1 and DC2 LSST DESC data challenge eras, that  enable a staged progression in dataset realism and analysis sophistication.
Twinkles 1 focuses on the problem of accurate light curve extraction in annual release (Level 2) data.
Its small dataset size but end-to-end nature make Twinkles useful as a ``pathfinder'' for the generation of other LSST DESC data challenge datasets.

\end{abstract}

% Keywords are ignored in the LSST DESC Note style:
\dockeys{latex: templates, papers: awesome}

\maketitlepost

% ----------------------------------------------------------------------
%

\section{Introduction}
\label{sec:intro}

Brief notes on end-to-end test philosophy. Where to publish and why.

Brief notes on infrastructure pathfinding.

Broad science goals for each DC era. Focus on Twinkles 1, with notes on Twinkles 2.

Scope and layout of this note.


% ----------------------------------------------------------------------

\section{Twinkles: a Tiny Simulated LSST Sky Survey}
\label{sec:concepts}

Twinkles concepts, including Twinkles 1 and 2 differences.


% ----------------------------------------------------------------------

\section{Science Analysis}
\label{sec:science}

Twinkles 1 science analysis plan, following from goals.

Brief discussion of Twinkles 2.




% ----------------------------------------------------------------------

\section{Survey Specifications}
\label{sec:survey}


% ----------------------------------------------------------------------

\section{Pipeline Development}
\label{sec:pipeline}

Twinkles 1 R\&D plan that we followed, so that people understand what Run 1.1 is etc).



Science-specific Tools:

SN and SL Sprinklers.

Monitor.




General-purpose Infrastructure:

Introductions to: PhoSim pipeline, DM Level 2 pipeline, Pserv.


% \begin{figure}
% \includegraphics[width=0.9\columnwidth]{example.png}
% \caption{An example figure: the LSST DESC logo, copied from \code{.logos/desc-logo.png} into \code{figures/example.png}. \label{fig:example}}
% \end{figure}


% ----------------------------------------------------------------------

\section{Discussion}
\label{sec:discussion}


% ----------------------------------------------------------------------

\section{Conclusions}
\label{sec:conclusions}

Here's a summary of what we just reported.

We can draw the following well-organized and neatly-formatted conclusions:
\begin{itemize}
  \item This is important.
  \item We can measure some number with some precision.
  \item This has some implications.
\end{itemize}

Here are some parting thoughts.


% ----------------------------------------------------------------------

\subsection*{Acknowledgments}

\input{acknowledgments}

%{\it Facilities:} \facility{LSST}

% Include both collaboration papers and external citations:
\bibliography{lsstdesc,main}

\end{document}
% ======================================================================
%
